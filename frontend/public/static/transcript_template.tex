\documentclass[a4paper,12pt]{article}
\usepackage[T1]{fontenc}
%\usepackage{serohyph}
\usepackage[ngerman]{babel}
\usepackage{latexsym}
\usepackage[pdfpagelabels,plainpages=false]{hyperref} %Für Übersichts-Lesezeichen im PDF
			% Fuer Print besser ’false’
\usepackage{setspace} % Zeilenabstand ändern mi
%\usepackage{sectsty} % section style package
\usepackage[%small
%,compact
]{titlesec}  % Paket um Überschriften einfach zu manipulieren
\usepackage{enumitem} % Paket für Auflistungen zu manipulieren (\setlist{noitemsep})
\usepackage{color} % für Farben

%\titleformat{ command }[ shape ]{ format }{ label }{ sep }{ before-code }[ after-code ]
\titleformat{\section} % titlesec
  {\normalfont\bfseries} %
  {\bfseries \thesection}
  {.5em}
  {}
\titlespacing{\section} % Um angehenden Textblock nach \section layouten.
             {0pc}{2ex}{3ex}[0pc] % Hiermit wird der Abstand zwischen Titel und Text verkelinert (=0)
%{0pc}{2ex}{0ex}[0pc] Abstand zum linken Rand |Abstand zum oberen Text | Abstand zum unteren Text | Abstand zum rechten Rand
%ACHTUNG letztes Klammerpaar eckige Klammern!!
%
\hypersetup{		% von package hyperref
colorlinks=true,        % Links farbig ('true') oder nicht ('false'). Fuer
                        % Print besser 'false'
linkcolor=red,          % Farbe fuer die internen Links
urlcolor=blue,          % Farbe fuer externe Links (http://...)
pdfborder={0 0 0},      % wenn colorlinks nicht gesetzt ist gibt es einen
                        % Rand dieser Farbe (R, G, B} um den Link
                        % geoeffnet werden
pdfpagemode=UseOutlines,% andere Moeglichkeiten: 'None', 'UseThumbs'
                        % und 'FullScreen'
pdftitle={Titel des Dokuments}, pdfauthor={Autor(en)},
pdfsubject={Thema}, pdfkeywords={Keywords} }

\addtolength{\textwidth}{2cm} %Textblcok-Breite
\addtolength{\oddsidemargin}{-1cm} %Abst. 1 inch von linkem Rand
\addtolength{\headsep}{-2cm} %Abst zu Kopfzeile
\addtolength{\textheight}{3cm} % Textblock-Länge
%
\def\headtitle{Example Course} % ToDo: Wiedergebbare Titel für den gesammten Text \headtitle
%
\title{\headtitle}
\author{VIS}
\makeindex

\begin{document}
\pagestyle{empty}
\parindent 0pt	% kein Erstzeileneinzug für das ganze Dokument
\flushleft % linksbündig
\setlist{noitemsep} %Kein Zeilenabstand zwischen Listen
%
%\maketitle
%
%\color{blue}
\color[rgb]{0,0,.5}
{\LARGE \headtitle\par}
\hrulefill\\
%
\section*{Oral Exam Report}
\vspace{.5cm}
%\begin{onehalfspace} %geht nur bei tabular, nicht bei "`tabbing"'
  \begin{tabbing}
    \quad \= Examiner:XXXX \= XXXXXXXXXXX \kill % Abstandgebende-Zeile
    \>Course:		\>	\headtitle\\[1ex]
    \>Examiner: 	\>	P. Muster\\[1ex]	%ToDo
    \>Protocol: 	\>	A. Muster\\[1ex]	%ToDo
    \>Semester:	\>	HS12\\[1ex]	%ToDo
    \>Datum: 		\>	2013-02-24%ToDo
  \end{tabbing}
%\end{onehalfspace}%
%
%\vspace{0.1cm}
\hrulefill\\
\color{black}
\vspace{.7cm}

%ToDo
\section{First question}
Lorem ipsum dolor sit amet, consectetur adipiscing elit. Curabitur lectus lorem, malesuada ac fringilla ac, sodales in ante. Pellentesque ultrices sodales felis. Sed sagittis tellus ut felis porta luctus. Ut adipiscing lobortis dui in dignissim. Praesent turpis orci, tincidunt nec volutpat et, hendrerit eu purus. Sed luctus tincidunt lobortis. Mauris et suscipit nisi. Pellentesque feugiat feugiat aliquet. Proin sed mauris nunc, a congue justo. Etiam semper placerat justo, non iaculis diam venenatis non. Integer ultrices quam id mi cursus id lobortis lacus molestie. Pellentesque habitant morbi tristique senectus et netus et malesuada fames ac turpis egestas. Morbi sem mi, tincidunt eu adipiscing consectetur, convallis luctus orci. Quisque vitae mattis urna. 
\section{Second question}
Donec id ante quis urna euismod vestibulum ut et dolor. Cras blandit venenatis urna at pharetra. Maecenas erat leo, volutpat a tempus eu, facilisis ac lorem. Fusce vitae felis vitae nibh interdum lacinia. Duis porttitor turpis nisi, laoreet dapibus enim. Vestibulum nec ligula orci, quis porta mi. Vivamus aliquet tellus quis lectus eleifend eget condimentum magna porta. 
\end{document}
